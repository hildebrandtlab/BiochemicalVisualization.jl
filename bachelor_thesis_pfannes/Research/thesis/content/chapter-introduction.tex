% !TEX root = ../my-thesis.tex
%
\chapter{Introduction}
\label{sec:intro}

\section{Motivation}
\label{sec:intro:motivation}

The ability to visualise biological structures and conduct simulations with them is a major breakthrough in the field of molecular biology. Scientists would commit themselves for several years studying a singular structure and construct a model for it by hand before applications for computational visualisation were developed \cite{OLSON20183997, 10.1038/nmeth.1427}. Not only was this a lengthy process, but it could also be hindersome for research until the molecule model had been finished. In order to accurately make predictions about the behaviour of molecules in certain situations like chemical reactions or protein-protein docking you need to understand their structure thoroughly first. After the first visualisation applications had been developed the fields of molecular biology and computational graphics continued to work together and enhanced the technologies used for structural analysis and synthesis \ref{OLSON20183997}. Nowadays, there is a vast amount of tools on the market used for not only molecular visualisation, but also other specialised tasks such as docking simulations for proteins, molecule editors for drug design and many more. The problem is that many of those applications only support basic visualisation functionalities such as different molecule models, basic interaction with molecules and loading biological structures from files without being able to be extended on the user's end if they need more specific features. This leads to development of tools that cover said features which in turn increases the multitude of applications to choose from. 

A solution to this problem would be to develop an application which not only provides basic visualisation and interaction functionalities, but also a way to extend the tool on the user's end. A piece of software that promises extensibility and ease-of-use is called BALLView \cite{10.1093/bioinformatics/bti818, 10.1007/s10822-005-9027-x}. BALLView was developed in C++ with an object-oriented and generic approach by Hildebrandt et al. It is a popular tool for molecule visualisation and modelling. Among the other functionalities that BALLView supports are the calculation of nuclear magnetic resonance spectra, search for structural similarities, molecular mechanics, solvent methods and support for importing or exporting files in various formats. Hildebrandt et al. added the possibility to extend the application by embedding scripts written in Python. Due to outdated dependencies and the majorities of the files being more than ten years old, Hildebrandt and his current research group at the Johannes Gutenberg University in Mainz decided to implement BALLView and its framework again from scratch using the Julia programming language. The project is called \textit{BiochemicalAlgorithms.jl} and aims for recreating the features of BALLView and transform it into a web-based visualisation application. The reason for that is to become platform independent. Instead of developing tools for specific Operating Systems (OS) you create applications that run in web browsers which in turn are available on every OS. Rendering big biological structures is also not a problem as web applications are able to leverage the Graphics Processing Unit (GPU) through the use of a library called WebGL \cite{BibEntry2011Jul}. 

The goal of our project is to provide the visualisation framework for \textit{BiochemicalAlgorithms.jl}, called \textit{Molecule Visualizer}. In order to keep it extensible and easy-to-use, we also follow an object-oriented approach. Communication between the different classes is managed by using events or object references if necessary. This way we ensure that functionalities will be grouped in their own respective class as well as decoupling objects by reducing dependencies between classes. As for basic features it supports visualisation of different molecule models like Ball and Stick, Wireframe or Space-Filling, loading of various molecule structure files and basic interaction involving picking atoms, transforming the molecule or changing atom colors. Instead of using WebGL as our main rendering engine we integrated a package called three.js \cite{BibEntry2022Oct} in our project. three.js is a general-purpose 3D library that supports renderer frameworks for WebGL, Spaced Vector Graphics (SVG) or Cascading Style Sheets 3D (CSS3D). Another major feature is the visualisation of molecules inside a Virtual Reality (VR) environment. The WebXR \cite{BibEntry2022Aug} Application Programming Interface (API) is used to create said environment by communicating with the VR device in order to render molecules on its display and leveraging the device's movement tracking capabilities for interactions with the VR environment. Our project can be found at the following URL: \url{https://gitlab.rlp.net/ppfannes/bachelor_thesis/-/tree/develop}.

\section{Thesis Structure}
\label{sec:intro:structure}

\textbf{Chapter \ref{sec:related}} \\[0.2em]
In this chapter we will provide an overview of important works that were relevant for our project. We include popular web-based tools that inspired us regarding functionalities that we implemented as well as discuss the advantages and disadvantages of web frameworks in general. Desktop software also play an important role in molecular research so we included some applications in order to highlight the differences between web-based and desktop tools.

\textbf{Chapter \ref{sec:visualconcepts}} \\[0.2em]
Important concepts of visualisation will be explained in this chapter as well as the main framework of our application, three.js. A demo program will be used throughout the chapter to explain how the different concepts are realised in three.js. Using the demo we also want to further illustrate our decision to use three.js in our project instead of WebGL.

\textbf{Chapter \ref{sec:implementation}} \\[0.2em]
The basic structure of our project and its core features will be discussed in this chapter. Using various code snippets we are going to explain the rendering process within our tool and how the different components interact with each other.

\textbf{Chapter \ref{sec:evaluation}} \\[0.2em]
In order to gain insight on how our tool fares against existing software we perform an an experiment to evaluate the various applications regarding two important aspects, user-friendliness and functionalities. The results of this experiment will give us the opportunity to categorise our project among other tools and gauge its utility to the bioinformatics community.

\textbf{Chapter \ref{sec:conclusion}} \\[0.2em]
In chapter six we will summarise the results of our project and give an outlook on what functionalities can be added in the future to the \textit{Molecule Visualizer}.
