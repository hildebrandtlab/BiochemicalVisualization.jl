% !TEX root = ../my-thesis.tex
%
\chapter{Related Work}
\label{sec:related}

The visualisation of molecules and larger proteins is an essential part of molecular biology. It provides the means to conduct further analysis of a molecule's structure and other experiments such as simulations of proteins docking together or drug design. Proper analysis also requires some form of interaction with the visualised object such as changing its representation or editing its structure. As such various tools have been developed over time to take on those tasks and help with research. Even though they all share the same core functionality, i.e. visualisation and basic interactions with the visualised object, the different tools offer a wide variety of intermediate functionality for analysing biological structures. Available software for molecular visualisation can be broadly categorised into two groups: desktop-based and web-based applications. 


\section{Web-based applications}
\label{sec:related:web}

Web-based visualisation tools offer the advantage that only a modern web browser, e.g. Mozilla Firefox or Google Chrome, is necessary in order to use them. This results in a great amount of flexibility as the applications run platform-independent. Deployment and testing of software becomes a much easier task as all you need is a web browser which can be easily installed on the OS of your choice. It is not necessary to emulate a specific OS to check for compatibility for your application. Another advantage is the direct support of mobile devices. Web-based molecule visualisation tools are in general JavaScript libraries that can be embedded into an Hyper Text Markup Language (HTML) web page. Thus, any device capable of displaying web pages can also use web-based visualisation tools. Among the popular choices for web-based molecule visualisation are MolView \cite{Bergwerf2022Oct}, 3Dmol.js \cite{10.1093/bioinformatics/btu829}, and BALLView.

MolView was first released in 2014 and is an open-source molecule visualisation and editing tool. It is based on JavaScript and utilises WebGL + HTML5 to render objects in a web browser. MolView offers a wide range of functionalities from editing options (e.g. add/remove atoms or bonds) over different molecule representations to miscellaneous options, provided through an embedding of JSmol, like molecule measurements, energy calculations and surface representations of electrostatic potentials. It is also possible to search for specific molecules in big databases like the RCSB Protein Data Bank, PubChem Compounds or the Crystallography Open Database by simply specifying the name of the molecule in question. However, biological structures cannot be imported from local files which can pose a problem if none of the used databases contain an entry for a molecule of interest. 

3Dmol.js, initially released in 2015, was developed in JavaScript only and specialises in the visualisation of molecules. There are several ways how 3Dmol.js can be used, either from a JavaScript file, as an embedded viewer in HTML or publicly as a hosted viewer. The diversity in use cases allows 3Dmol.js to be used by not only programmers, but also web developers or even people without programming background. Molecules can be loaded either from local files or by inputting their ID within the PDB. Hardware acceleration as well as parallelised computation of molecular surfaces allow the user to efficiently view and interact with even complex scenes. The developers chose an object-oriented approach for 3Dmol.js so that it can be used in combination with other tools that provide more chemical information regarding the visualised molecule.

Our goal is to provide a visualisation tool that will be integrated into the Biochemical Algorithms Library (BALL) by Andreas Hildebrandt et al. BALL's development started in 1996 as a tool for protein-protein docking, but it quickly grew into a much bigger application supporting various functionalities such as molecule and surface visualisation, loading of multiple molecules at once, several molecular representations and modelling features. All these features are packed into their standalone viewer based on BALL, BALLView. The main goal of the developers was to create a robust and easy-to-use tool for visualising biochemical structures and drug design. For this purpose the application was designed to implement all the functionalities necessary to properly support users during their research. However, they also made sure that it is possible to easily extend the software in case more specified features are needed by the user. Currently, Andreas Hildebrandt and his team are trying to completely reimplement BALL in Julia as the original software depends on old and outdated libraries like OpenGL 1.3. The project is called \textit{BiochemicalAlgorithms.jl} and can be found at the following URL: \url{https://github.com/hildebrandtlab/BiochemicalAlgorithms.jl}.

\section{Desktop applications}
\label{sec:related:desktop}

Desktop software runs directly on the OS of your system which makes them powerful tools. Through OpenGL and the use of shaders the tools communicate with graphics hardware in order to render objects. Operating directly on the OS enables visualisation software to access the various components of the PC and thus leveraging them to address even complex tasks such as simulations or the calculation of molecular surfaces. The major downside to this approach though is the inflexibility of the software. Support for multiple OS requires additional efforts from the developers which could be used to enhance the quality of the product. Additionally, users might not be able to use the software at all if it is not available for their OS which further complicates researches.

Jmol \cite{BibEntry2022Mar} is an open-source Java application for viewing and editing molecules. It started development in the early 2000s and has become one of the most used visualisation tools. It does not depend on WebGL or hardware to render molecules on screen but rather on a framework solely designed for Jmol called z-buffer. To integrate Jmol into other Java applications or web browsers the developers created dedicated components: JSmol which allows for integration into web browsers and JmolViewer which can be embedded in other applications. It also features cross-platform support for Windows, Mac OS X and Linux/Unix systems. Jmol supports a wide range of data formats molecules can be loaded from as well as functionalities for animations, molecule measurements, surfaces and many more.

One of the most used desktop applications for molecule visualisation is PyMOL \cite{PyMOL}. It was initially released in 2000 by Warren Lyford DeLano as an open-source project under the Python License. His ambition was to demonstrate the impact of open-source software on scientific researches by distributing the source code for PyMOL publicly. After his death, Schrödinger Inc. acquired PyMOL in 2010 and since then it has grown into a system with various features not only for molecule visualisation, but also for simulations or drug design. Among different molecule representations, calculation of surface models and tools for animations there is also the possibility to load biochemical structures from local files or via their molecule ID. Instead of a menu bar at the top of the application window PyMOL offers a second window that acts as a command line tool. From there you can perform your manipulations of the displayed object in the main window. The Python interface of PyMOL supports the deployment of additional plug-ins and execution of scripts.