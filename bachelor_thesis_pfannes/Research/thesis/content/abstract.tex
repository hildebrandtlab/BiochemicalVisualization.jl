% !TEX root = ../my-thesis.tex
%
\pdfbookmark[0]{Abstract}{Abstract}
\addchap*{Abstract}
\label{sec:abstract}

Molecule visualisation software is widely used in the research of biological structures. They offer various models for representing structures as well as means of interacting with them. There is a wide variety of tools available which specialise on certain aspects of molecular research in the form of desktop applications, but also as web-based visualisation software. The main problem of those tools is that the majority of them are not extensible on the user's end. If they need features the tool which they currently use cannot offer they have to find a new application. This results in users having to learn several tools at once. In this thesis we present the \textit{Molecule Visualizer}, a visualisation framework that is supposed to be embedded into \textit{BiochemicalAlgorithms.jl}, a project aiming to re-implement the functionality of BALLView inside a web-based application. The \textit{Molecule Visualizer} provides functionality for visualising molecules in different representations as well as interacting with them. Due to its object-oriented nature we laid the foundation of an extensible visualisation interface in case users need additional functionalities.

\vspace*{20mm}

{\usekomafont{chapter}Abstract}
\label{sec:abstract-diff}

Molekulare Visualisierungssoftware ist weit verbreitet bei der Untersuchung von biologischen Strukturen. Sie bieten die Möglichkeit, Strukturen in verschiedenen Repräsentationen zu betrachten sowie mit ihnen zu interagieren. Es existiert eine große Vielfalt and Software sowohl als Desktopapplikation als auch web-basierte Software, die sich auf bestimmte Bereiche der molekularen Forschung spezialisiert haben. Deren Hauptproblem jedoch ist, dass sie sich auf Seiten der Benutzer nich erweitern lassen. Falls ein Benutzer weitere Funktionalität benötigt, dann muss sich dieser eine neue Applikation suchen. Das hat zur Folge, dass sich Benutzer unter Umständen mit mehreren Tools auf einmal auseinandersetzen müssen. In dieser Arbeit präsentieren wir den \textit{Molecule Visualizer}, ein Visualisierungsframework welches in \textit{BiochemicalAlgorithms.jl} integriert werden soll. Der \textit{Molecule Visualizer} stellt Funktionalitäten zur Verfügung, um Moleküle in verschiedenen Repräsentationen darzustellen sowie mit ihnen zu interagieren. Aufgrund seiner objektorientierten Natur bietet er eine gute Grundlage, um auf Seiten der Benutzer erweitert zu werden, falls weitere Funktionalität benötigt wird, die \textit{BiochemicalAlgorithms.jl} nicht bietet.