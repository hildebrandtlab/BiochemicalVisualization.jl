% !TEX root = ../my-thesis.tex
%
\chapter{Conclusion}
\label{sec:conclusion}

We presented the \textit{Molecule Visualizer}, a visualisation framework that is capable of rendering molecules in different models and offers basic interaction with the visualised structures. We implemented it as a web-based visualisation tool, because it is easier to exchange research results and the performance of rendering in web browsers has drastically increased with the introduction of WebGL. In order to render objects we used the three.js library instead of WebGL directly, because three.js offers an intuitive API that simplifies mesh creation and rendering. It also provides the means to quickly set up a VR environment without many changes to the render loop. 

Our software is meant to be to be integrated into the \textit{BiochemicalAlgorithms.jl} project by Hildebrandt et al. As they envision their project to be easy-to-use and extensible on the user's end, we tried to design our project around those ideas. We followed an object-oriented approach that encapsulates functionalities in their own classes, e.g. the \textit{ObjectPicker} class for picking objects or the \textit{Model} class for managing molecule models. Reducing dependencies between objects is realised using communication based on events sent between classes. The advantage of an object-based design is that users are later able to add new features by defining new objects which inherit from our base classes. Those objects would then implement the new functionalities the user needs. 

In the end we evaluated our project alongside other visualisation tools regarding their user-friendliness and the different functionalities they support. The \textit{Molecule Visualizer} does not support as many functionalities as the other tools, but it still has a solid foundation which leaves room for improvement. We talk about further additions in section \ref{sec:conclusion:future}. It is also one of the few tools that is built towards extensibility together with PyMOL which supports Python scripts with additional functionalities. Following an object-oriented approach makes it easier for users to either define completely new objects and add them to the \textit{Molecule Visualizer} or add more features to already existing objects. It also helps with integrating our tool into \textit{BiochemicalAlgorithms.jl}. The user-friendliness of our application compared to the other tools is above average. While some projects halted development or do not offer a well-structured documentation the Molecule Visualizer is still being developed and also offers documentation in form of a separate web page. In the future we will address various bugs that could not be solved due to time constraints and add more features to make our tool even more versatile.

\section{Future Work}
\label{sec:conclusion:future}

The features of our \textit{Molecule Visualizer} can be extended by making the following additions:
\begin{itemize}
\item Due to time constraints our project could not be integrated into \textit{BiochemicalAlgorithms.jl} yet. If it were integrated we would be able to add more molecule models like Cartoon. This model gives biological structures a comic-like appearance and is a good choice for visualising proteins. We already added a prototype for the Cartoon model setting which uses Catmull-Rom splines to render a tube given an array of control points. We would replace the current control points with an array containing the positions of $\alpha$-carbon atoms which form the backbone of a protein. The spline would then run through the backbone of the protein, yielding the desired cartoon representation.
\item Currently it is only possible to visualise and interact with molecules inside VR, but there is no GUI for manipulating its appearance or changing models. This is a limitation of the WebXR API. It is not possible to render interactive HTML/CSS directly in a VR scene. This is why we resorted to three-mesh-ui for the modal boxes that appear when you pick an atom. three-mesh-ui only supports a small amount of interactive elements though e.g. buttons or a virtual keyboard. There are no drop-down menus or checkboxes that we use for selecting different models and providing test molecules for visualisation. But there are other alternatives in development like CanvasUI which supports different shapes for textboxes, scrolling text and color selection. Drop-down menus and sliders are in the works. 
\item Another important feature of BALLView besides visualisation is the ability to build peptides. This functionality is especially helpful for drug design where you want to create new active ingredients to cure illnesses. This can be done by changing the instance count of the Instanced Meshes to add or remove atoms and bonds. \textit{BiochemicalAlgorithms.jl} would have to take on the task of making sure the resulting molecule is physically correct, i.e. the bond lengths and angles between the atoms are correct.
\end{itemize}